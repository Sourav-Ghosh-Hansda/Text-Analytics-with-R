% Options for packages loaded elsewhere
\PassOptionsToPackage{unicode}{hyperref}
\PassOptionsToPackage{hyphens}{url}
%
\documentclass[
]{article}
\usepackage{amsmath,amssymb}
\usepackage{lmodern}
\usepackage{iftex}
\ifPDFTeX
  \usepackage[T1]{fontenc}
  \usepackage[utf8]{inputenc}
  \usepackage{textcomp} % provide euro and other symbols
\else % if luatex or xetex
  \usepackage{unicode-math}
  \defaultfontfeatures{Scale=MatchLowercase}
  \defaultfontfeatures[\rmfamily]{Ligatures=TeX,Scale=1}
\fi
% Use upquote if available, for straight quotes in verbatim environments
\IfFileExists{upquote.sty}{\usepackage{upquote}}{}
\IfFileExists{microtype.sty}{% use microtype if available
  \usepackage[]{microtype}
  \UseMicrotypeSet[protrusion]{basicmath} % disable protrusion for tt fonts
}{}
\makeatletter
\@ifundefined{KOMAClassName}{% if non-KOMA class
  \IfFileExists{parskip.sty}{%
    \usepackage{parskip}
  }{% else
    \setlength{\parindent}{0pt}
    \setlength{\parskip}{6pt plus 2pt minus 1pt}}
}{% if KOMA class
  \KOMAoptions{parskip=half}}
\makeatother
\usepackage{xcolor}
\usepackage[margin=1in]{geometry}
\usepackage{graphicx}
\makeatletter
\def\maxwidth{\ifdim\Gin@nat@width>\linewidth\linewidth\else\Gin@nat@width\fi}
\def\maxheight{\ifdim\Gin@nat@height>\textheight\textheight\else\Gin@nat@height\fi}
\makeatother
% Scale images if necessary, so that they will not overflow the page
% margins by default, and it is still possible to overwrite the defaults
% using explicit options in \includegraphics[width, height, ...]{}
\setkeys{Gin}{width=\maxwidth,height=\maxheight,keepaspectratio}
% Set default figure placement to htbp
\makeatletter
\def\fps@figure{htbp}
\makeatother
\setlength{\emergencystretch}{3em} % prevent overfull lines
\providecommand{\tightlist}{%
  \setlength{\itemsep}{0pt}\setlength{\parskip}{0pt}}
\setcounter{secnumdepth}{-\maxdimen} % remove section numbering
\ifLuaTeX
  \usepackage{selnolig}  % disable illegal ligatures
\fi
\IfFileExists{bookmark.sty}{\usepackage{bookmark}}{\usepackage{hyperref}}
\IfFileExists{xurl.sty}{\usepackage{xurl}}{} % add URL line breaks if available
\urlstyle{same} % disable monospaced font for URLs
\hypersetup{
  pdftitle={MA331-Report: 2211606},
  pdfauthor={Ghosh Hansda, Sourav},
  hidelinks,
  pdfcreator={LaTeX via pandoc}}

\title{MA331-Report: 2211606}
\usepackage{etoolbox}
\makeatletter
\providecommand{\subtitle}[1]{% add subtitle to \maketitle
  \apptocmd{\@title}{\par {\large #1 \par}}{}{}
}
\makeatother
\subtitle{TED Talks by Speaker Alan Eustace and Speaker Tshering Tobgay}
\author{Ghosh Hansda, Sourav}
\date{}

\begin{document}
\maketitle

\hypertarget{introduction}{%
\subsection{Introduction}\label{introduction}}

This report is an exploratory analysis of two TED talks, one each by
Alan Eustace and Tshering Tobgay. Alan Eustace's speech took place on
October 21, 2014. He is a former Google executive. Tshering Tobgay
delivered the speech on April 1, 2016. He is a Bhutanese politician who
served as the Prime Minister of Bhutan from 2013 to 2018. The objective
of this report is to identify common themes in their speeches and
conduct a sentiment analysis of the words uttered.

\hypertarget{methods}{%
\subsection{Methods}\label{methods}}

The data used in this analysis is from the ted\_talks dataset available
in the dsEssex package in R. The dataset contains information about
various TED Talks, including the speaker, the title of the talk, and the
transcript of the speech.The data preparation process involves loading
the required packages, dsEssex, tidyverse, tidytext, ggplot2, and
ggrepel. So the dataset is filtered to include speeches by the two
speakers and then tokenize the text using the ``unnest\_tokens()''
function from the ``tidytext'' package. The analysis removes stop words,
as well as some words like ``laughter'' and ``applause,'' to create a
word count table for each speaker. Visualizing the top 25 words used by
each speaker is done by using the ``slice\_max()'', ``mutate()'', and
``geom\_col()'' functions from the ``ggplot2'' package. To compare the
top words used by both speakers, the report uses ``bind\_rows()'',
``group\_by()'', ``filter()'', ``pivot\_wider()'', and
``geom\_abline()'' functions from ``ggplot2'' and labels the words using
``geom\_text\_repel()'' function from ``ggrepel''. Performing sentiment
analysis uses the NRC lexicon. It uses the ``get\_sentiments(''nrc'')''
function to obtain the NRC lexicon and joins the lexicon with the
tokenized text using ``inner\_join()''. The analysis counts the number
of words associated with each sentiment for each speaker using
``count()'', and pivots the data using ``pivot\_wider()'' to create a
table with columns for each speaker and rows for each sentiment. It
calculates the odds ratio and log odds ratio of each sentiment for each
speaker using ``mutate()''. The odds ratio is calculated as the ratio of
the number of words associated with a sentiment to the number of words
not associated with the sentiment. The log odds ratio is calculated as
the natural logarithm of the odds ratio.The most common positive and
negative sentiment lexicons are extracted from a dataset of talks.
First, the unnest\_tokens() function is used to separate the text into
individual words. The anti\_join() function is then used to remove stop
words, followed by the inner\_join() function to match the remaining
words with the National Research Council (NRC) sentiment lexicon, which
classifies words as positive, negative, or neutral. The resulting words
are then filtered to select only those with positive or negative
sentiment, and their frequency is counted using count(). Two plots are
created to show the most common positive and negative words. The
slice\_max() function is used to select the top 10 words, and the
fct\_reorder() function is used to order the words by frequency. The
plots are created using ggplot(), with geom\_col() to create a bar plot,
and coord\_flip() to flip the axes. The x- and y-labels are set using
xlab() and ylab(), and the plot titles are set using ggtitle(). Finally,
the plots are displayed side by side using grid.arrange().

\hypertarget{results}{%
\subsection{Results}\label{results}}

The tables of Alan Eustace and Tshering Tobgay show the words they used
most frequently in their respective talks.Here Alan has used maximum the
word `parachute' 15 times whereas Tshering has used `carbon' 23 times.

\begin{verbatim}
# A tibble: 476 x 3
   speaker      word             n
   <chr>        <chr>        <int>
 1 Alan Eustace parachute       15
 2 Alan Eustace can             14
 3 Alan Eustace right           12
 4 Alan Eustace one             10
 5 Alan Eustace see             10
 6 Alan Eustace going            9
 7 Alan Eustace stratosphere     9
 8 Alan Eustace suit             9
 9 Alan Eustace balloon          8
10 Alan Eustace system           8
# ... with 466 more rows
\end{verbatim}

\begin{verbatim}
# A tibble: 610 x 3
   speaker         word         n
   <chr>           <chr>    <int>
 1 Tshering Tobgay carbon      23
 2 Tshering Tobgay country     22
 3 Tshering Tobgay bhutan      15
 4 Tshering Tobgay neutral     12
 5 Tshering Tobgay together    11
 6 Tshering Tobgay one         10
 7 Tshering Tobgay us          10
 8 Tshering Tobgay world       10
 9 Tshering Tobgay years       10
10 Tshering Tobgay change       9
# ... with 600 more rows
\end{verbatim}

The first two bar graphs depict the top 25 words used by both the
speakers.Alan has mostly been using words like `parachute',`can' \&
`right' whereas Tshering has mostly been using words like
`carbon',`country' \& `bhutan'.The third figure shows a scatter plot
where each pair of words is represented by a point on the plot, with the
x-axis representing the frequency of the word in Alan Eustace's speech,
and the y-axis representing the frequency of the word in Tshering
Tobgay's speech. The line of best fit shows the trend in the data.

\includegraphics{mark1_files/figure-latex/unnamed-chunk-2-1.pdf}
\includegraphics{mark1_files/figure-latex/unnamed-chunk-2-2.pdf}

The table below shows which emotions were expressed more frequently by
each speaker during the talks.The values in the table show the number of
words associated with each emotional tone for each speaker.Here Tshering
is showing more trust than Alan \& having a more positive \& joyous
tone.

\begin{verbatim}
# A tibble: 10 x 3
   sentiment    `Alan Eustace` `Tshering Tobgay`
   <chr>                 <int>             <int>
 1 anger                     3                16
 2 anticipation             36                68
 3 disgust                   3                10
 4 fear                     29                33
 5 joy                      24                65
 6 negative                 27                34
 7 positive                 71               151
 8 sadness                  17                10
 9 surprise                 20                30
10 trust                    42                96
\end{verbatim}

The bar chart below displays the log odds ratio for each sentiment, with
sentiments ordered based on their association with the speakers.
Sentiments that are more strongly associated with Alan Eustace appear on
the left side of the chart, while those more strongly associated with
Tshering Tobgay appear on the right side. The color of the bars
indicates whether the log odds ratio is positive (dark green) or
negative (red). The x-axis shows the log odds ratio, which is a
statistical measure that reflects the strength and direction of the
association between the sentiment and the speakers. A lower log odds
ratio indicates a stronger association with Alan Eustace, while a higher
log odds ratio indicates a stronger association with Tshering Tobgay.
Overall, the visualization helps to identify which sentiments are more
strongly associated with each speaker and to compare the differences in
sentiment expression between them.

\includegraphics{mark1_files/figure-latex/unnamed-chunk-4-1.pdf}

In the following chart, we have displayed the most common positive words
and most common negative words used in both the talks.The words like
`laughter' \& `applause' have more postive impact \& on the other hand
words like `government', `impossible' \& `fight' are having a more
negative impact.

\includegraphics{mark1_files/figure-latex/unnamed-chunk-5-1.pdf}

\end{document}
